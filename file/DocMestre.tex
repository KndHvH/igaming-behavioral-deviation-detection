
\documentclass[a4paper, 12pt, oneside]{memoir}

\usepackage[top=3cm, left=3cm, right=3cm, bottom=3cm]{geometry}

\usepackage[protrusion=true, final, factor=1100]{microtype}
\usepackage[brazil, english]{babel}


\usepackage{fontspec}
\setmainfont[Ligatures=TeX, 
Numbers={Proportional, OldStyle}
]{Linux Libertine O}
\setsansfont{Linux Biolinum O}


% Bibliografia com estilo Autor (Ano)
\usepackage[citestyle=authoryear, sorting=nyt, style=authoryear]{biblatex}

% Se preferir o estilo de bibliografia ABNT, use a linha a seguir:
% \usepackage[citestyle=authoryear, sorting=nyt, style=abnt]{biblatex}  
\addbibresource{./referencias/bibliografia.bib}
\setlength{\bibitemsep}{0.5\baselineskip}  % Aumenta o espaço entre as obras nas Referências

\usepackage[usenames, dvipsnames]{xcolor}
\usepackage{graphicx}
\usepackage{algorithm}
    \renewcommand{\listalgorithmname}{Lista de Algoritmos}
\usepackage{algpseudocode}

\usepackage{minted}  % P/ scripts em Python (ou qualquer outra linguagem)
\usepackage{neuralnetwork}  % P/ desenhar Redes Neurais simples

\usepackage{caption}
    \captionsetup{margin=10pt, font=footnotesize, labelfont=bf}
\usepackage[all]{nowidow}  % Controle de linhas viúvas e órfãs
\RequirePackage{csquotes}  % Citações de acordo com o padrão da língua

\usepackage[pdfencoding=auto, psdextra, unicode=true, 
colorlinks=true, urlcolor=blue, citecolor=blue, linkcolor=brown]{hyperref}

%%%% Formatação do Sumário %%%%%
\renewcommand*{\cftchapterleader}{}
\renewcommand*{\cftsectionleader}{}
\renewcommand*{\cftsubsectionleader}{}
\renewcommand{\cftchapterpagefont}{}
\renewcommand*{\cftchapterformatpnum}[1]{~~\textbullet~~\textit{#1}}
\renewcommand*{\cftsectionformatpnum}[1]{~\textbullet~#1}
\renewcommand*{\cftsubsectionformatpnum}[1]{~\textbullet~#1}
\renewcommand{\cftchapterafterpnum}{\cftparfillskip}
\renewcommand{\cftsectionafterpnum}{\cftparfillskip}
\renewcommand{\cftsubsectionafterpnum}{\cftparfillskip}
\setrmarg{3.55em plus 1fil}
\setsecnumdepth{subsection}
\maxsecnumdepth{subsection}
\settocdepth{subsection}

\setlength\cftsectionindent{2em}
\setlength\cftsubsectionindent{4em}
\setlength\cftchapternumwidth{2em}
\setlength\cftsectionnumwidth{2em}
\setlength\cftsubsectionnumwidth{2em}

% Para centralizar cabeçalhos nas tabelas:
\newcommand*{\centralizar}[1]{\multicolumn{1}{c}{\bfseries #1}}


%%%%%%%%%%%%%%%%%%%%%%%%%%%%%%%%%%%%%%%%%
\begin{document}

\selectlanguage{brazil}
\pagestyle{empty}

% %%%% 
% VOCÊ DEVE ABRIR O ARQUIVO InformacoesCatalograficas.tex E
% PREENCHÊ-LO COM AS INFORMAÇÕES RELATIVAS AO SEU TRABALHO. 
% %%%%
%%%%%%%%%%%%%%%%%%%%%%%%%%%%%%%%%%%%%%%%%%%%%%%%%%%%%%%%%%%%%%
% PREENCHA ABAIXO COM AS INFORMAÇÕES NECESSÁRIAS
%%%%%%%%%%%%%%%%%%%%%%%%%%%%%%%%%%%%%%%%%%%%%%%%%%%%%%%%%%%%%%

%%% Título 
\newcommand{\TítuloDoTrabalho}{%
  Título do trabalho apenas com a primeira letra em maiúscula, sem ponto final}

%%% Título em inglês
\newcommand{\TítuloEmIngles}{%
  Título do trabalho (em inglês) apenas com a primeira letra
  em maiúscula, sem ponto final}

%%% Versão original ou corrigida? Escolha.
\newcommand{\Versao}{Versão 
Original
% Corrigida
}

%%% Autor (nome completo por extenso)
\newcommand{\Autor}{%
  Nome completo do autor 
}

%%% Qual é seu último nome? (Como você usa nas publicações). Se houver
%%% Júnior, Filho, etc, indique apropriadamente. Ex: Silva Filho.
\newcommand{\Sobrenome}{%
  Tal}

%%% Qual é seu primeiro nome, nome do meio, e demais partes do
%%% sobrenome (não incluídas acima)? A soma deste item e do anterior
%%% devem formar seu nome completo por extenso. Não abrevie.
\newcommand{\Nome}{%
  Fulano de}

%%% Escreva as iniciais do seu nome *sem o sobrenome* com a pontuação correspondente,
%%% tal como você quer ver na Ficha Catalográfica
\newcommand{\IniciaisDoNome}{%
  F.}


%%%% Indique o nome do seu Orientador (nome completo por extenso)
\newcommand{\Orientador}{%
  Nome completo do orientador}

%%% Seu orientador(a) é Prof. Dr. ou Profa. Dra.? Selecione umas das
%%% linhas abaixo. Não remova as barras invertidas, apenas selecione a
%%% linha apropriada, comentando a outra
\newcommand{\DoutorOuDoutora}{%
  Prof.\ Dr.\
  %Profa.\ Dra.\
}

%%%% Dizeres da titulação
\newcommand{\TítuloObtido}{%
  Monografia apresentada ao Programa de Educação Continuada em Engenharia da Escola Politécnica da Universidade de São Paulo como parte dos requisitos para conclusão do curso de Especialização em Inteligência Artificial.}


%%%% Informe o ano de depósito do trabalho
\newcommand{\AnoDepósito}{%
  2021}

%%%% Indique o número total de páginas do trabalho
\newcommand{\NumPáginas}{%
  150}

%%%% Tipo de trabalho
\newcommand{\TipoTrabalho}{%
  Monografia (Especialização em Inteligência Artificial)}


%%%% Indique as palavras-chave para a FICHA CATALOGRÁFICA
% Cada uma delas deve ser precedida por um número com ponto.
% Todas palavras devem iniciar com letras maiúsculas
\newcommand{\PalavrasChaveFicha}{%
  1.~Palavra-Chave1 ~~2.~Palavra-Chave2 ~~3.~Palavra-Chave3}


%%% Palavras chave para Resumo e Abstract

% Resumo
% Use a mesma lista indicada acima
% Todas palavras devem iniciar com letras maiúsculas e terminar com um ponto final.
\newcommand{\PalavrasChave}{%
  Palavra-Chave1. Palavra-Chave2. Palavra-Chave3.}

%%%% Palavras-chave em inglês.
% Siga as regras para as palavras-chave em português.
\newcommand{\Keywords}{%
  Keyword1. Keyword2. Keyword3,}
 

% Capa
\begin{titlingpage}
\pagestyle{empty}

%%%%%%%%%%
% Capa 

\sffamily
\begin{center}

\textsc{Universidade de São Paulo} \\
\textsc{Escola Politécnica} \\
\textsc{Programa de Educação Continuada em Engenharia} \\
\textsc{Especialização em Inteligência Artificial} \\

\vspace{0.2\textheight}
\large
\Autor

% Título
\bigskip\bigskip
{\Large \bfseries
  \TítuloDoTrabalho
}

\vfill

{\normalsize 
São Paulo\\
\AnoDepósito}

\end{center}
\clearpage



%%%%%%%%%%
% Folha de rosto

\begin{center}
{\large \lsstyle \MakeUppercase{\Autor}}

\vspace{0.2\textheight}
{\Large \bfseries
  \TítuloDoTrabalho \\
}
\normalsize
\bigskip
--- \Versao ---

\vspace{0.2\textheight}
\begin{flushright}
\begin{minipage}{0.5\textwidth}
    \TítuloObtido

    \bigskip
    Orientador: \DoutorOuDoutora \Orientador
\end{minipage}
\end{flushright}

\vfill
São Paulo\\
\AnoDepósito

\end{center} 
\clearpage
\end{titlingpage}


\frontmatter
    \pagestyle{plain}

% Ficha catalográfica
\baselineskip 0.7\onelineskip
\footnotesize \sffamily

\begin{center}
    Autorizo a reprodução e divulgação total ou parcial deste trabalho, por qualquer meio \\ convencional ou eletrônico, para fins de estudo e pesquisa, desde que citada a fonte.

    \vfill

    Catalogação-na-publicação

    \bigskip

      \fbox{%
      \begin{minipage}{0.7\textwidth}
          \Sobrenome, \Nome
    
          \hspace{2em}\TítuloDoTrabalho / \IniciaisDoNome \Sobrenome{} -- São Paulo,
          \AnoDepósito.\\
          \hspace*{2em}\NumPáginas p.
          
        \bigskip
        \hspace{2em}\TipoTrabalho \, -- Escola Politécnica da Universidade de São Paulo. PECE -- Programa de Educação Continuada em Engenharia.
    
        \bigskip
        \hspace{2em}\PalavrasChaveFicha. \\
        I. Universidade de São Paulo. Escola Politécnica. PECE -- Programa de Educação Continuada em Engenharia. 
        II.t.
        
      \end{minipage}
      } % fbox
\end{center}

\normalfont \normalsize
\clearpage

% Dedicatória (opcional)

~
\vfill
Para Fulano, importante na minha vida.
\vfill

   
\clearpage

% Agradecimentos (opcional)
\chapter*{Agradecimentos}



%% Insira aqui seus Agradecimentos. Não esqueça de mencionar as Agências
%% Financiadoras.

\vfill
Sicrano, pelo auxílio na redação.
\vfill
\clearpage

% Epígrafe (opcional)

~
\vfill

\begin{center}
    \begin{minipage}{10cm}
        \begin{verse}
        E, se bem que seja obscuro\\
        Tudo pela estrada fora,\\
        E falso, ele vem seguro,\\
        E, vencendo estrada e muro,\\
        Chega onde em sono ela mora.\\
        E, inda tonto do que houvera,\\
        A cabeça, em maresia,\\
        Ergue a mão, e encontra hera,\\
        E vê que ele mesmo era A Princesa que dormia.
        \end{verse}
    \end{minipage}
\end{center}

\begin{flushright}
    \textit{--- Fernando Pessoa}
\end{flushright}
\clearpage

% Sumário
\abstractintoc
\tableofcontents
\clearpage

\input{./pre-textual/Resumo.tex}
\clearpage

\input{./pre-textual/Abstract.tex}
\clearpage

\listofalgorithms
\clearpage

\listoffigures
\clearpage

\listoftables
\clearpage



%%%%%%%%%%%%%%%%%%%%%%%%%%%%%%%%%%%%%%%%%%%%%%%%%%%

\mainmatter

\chapterstyle{madsen} % Algumas alternativas: ell; bianchi; demo2; veelo

    \pagestyle{companion}
    \OnehalfSpacing
    
    \chapter{Introdução}
\label{cap:Introducao}

Este capítulo deve:

\begin{itemize}
    \item Apresentar o problema que você se propôs a estudar;
    \item Situar seus objetivos diante desse problema;
    \item Justificar a escolha do tema, isto é, mostrar sua pertinência acadêmica ou social. Faz parte da justificativa acadêmica a identificação de lacunas nos trabalhos já publicados sobre o assunto, como quando ninguém parece ter tratado de determinado problema com determinado método (que é o que você propõe);
    \item Delimitar o seu trabalho: diante de tais e tais possibilidades de tratar do problema, quais delas serão efetivamente realizadas por você? 
    \item Expor sucintamente o conteúdo dos capítulos seguintes.
\end{itemize}
    \chapter{Revisão da literatura}
\label{cap:Revisao}

Neste capítulo, você deve discutir criticamente outros trabalhos relacionados ao tema de pesquisa escolhido.

Não se trata de simplesmente citar a existência de textos, mas de destacar contribuições centrais trazidas por eles em termos de métodos (modelos e conjuntos de dados utilizados), resultados, incluindo a apresentação do estado da arte, as dificuldades encontradas pelos autores e as diferenças mais importantes entre o seu trabalho e os outros.


\section{Exemplos de citações}

Referências de livros devem exibir as páginas citadas \parencite[ver][p.~10]{russell2002artificial}. Referências de artigos vão sem números de páginas, como \textcite{vaswani2017attention}.
    \chapter{Métodos}
\label{cap:Metodos}

Este capítulo deve apresentar as técnicas de investigação usadas na pesquisa. Lembre-se: quem lê o seu trabalho tem de ser capaz de replicá-lo seguindo as indicações presentes neste capítulo.


\section{Procedimentos}
\label{sec:Procedimentos}

Como foi feita a pesquisa? Aqui devem ser descritas as técnicas usadas:

\begin{itemize}
    \item Algoritmos
    \item Pipelines de algoritmos (se for o caso)
    \item Códigos de implementação, se esses detalhes forem essenciais  (raramente são!)
    \item Modelos e técnicas usados (exemplos: classificadores, tipos de redes neurais etc.)
    \item Testes de inferência estatística (se for o caso)
\end{itemize}


\section{Materiais}
\label{sec:Materiais}

De onde vêm os seus dados? Você deve não somente nomear seus conjuntos de dados, apresentando seus autores e páginas Web para acesso, como também descrever a natureza dos dados: são abertos ao público? Qual a fonte usada pelos autores para gerar o conjunto de dados?


\section{Instrumentos}
\label{sec:Instrumentos}

O que você usou para analisar os dados? Descreva brevemente o hardware e o software utilizados.


\section{Alguns exemplos}

\subsection{Algoritmos}

Há muitos pacotes para criar algoritmos. Vou apresentar o que me parece o mais simples de usar: algorithmicx. 

\bigskip  % Força um salto de linha. Use com cuidado!

\begin{algorithmic}
\If {$i\geq maxval$}
    \State $i\gets 0$
\Else
    \If {$i+k\leq maxval$}
        \State $i\gets i+k$
    \EndIf
\EndIf
\end{algorithmic}

\bigskip 

Lembre-se de que tudo o que aparece entre \$ está no modo matemático.

É muito útil colocar os algoritmos dentro do ambiente flutuante \textit{algorithm}. Como no caso dos outros flutuantes, ele fará a melhor escolha quanto ao posicionamento do bloco do algoritmo na página, de acordo com a otimização do fluxo do texto, inclusive colocando um bom espaço de distância entre o ambiente e o texto, evitando a necessidade de forçar saltos de linha como fizemos com \textit{bigskip}. O ambiente traz, ainda, uma forma padronizada de apresentar a legenda com o nome do algoritmo. Por fim, permite gerar uma lista dos algoritmos usados no texto. Veremos esse recurso em funcionamento quando falarmos da classe de documento \textit{book}.

\begin{algorithm}[ht]
\caption{Euclid’s algorithm}\label{euclid}
\begin{algorithmic}[1]
    \Procedure{Euclid}{$a,b$}\Comment{The g.c.d. of $a$ and $b$}
        \State $r\gets a\bmod b$
        \While{$r\not=0$}\Comment{We have the answer if $r$ is 0}
            \State $a\gets b$
            \State $b\gets r$
            \State $r\gets a\bmod b$
        \EndWhile\label{euclidendwhile}
        \State \Return $b$\Comment{The gcd is $b$}
    \EndProcedure
\end{algorithmic}
\end{algorithm}

Ah, você quer mudar as palavras-chave do inglês para o português? É simples:

\algrenewcommand\algorithmicwhile{\textbf{Enquanto}}
\algrenewcommand\algorithmicdo{\textbf{faça}}
\algrenewcommand\algorithmicend{\textbf{Fim}}
\algrenewcommand\algorithmicprocedure{\textbf{Função}}
\algrenewcommand\algorithmicfor{\textbf{Para}}
\algrenewcommand\algorithmicif{\textbf{Se}}
\algrenewcommand\algorithmicthen{\textbf{Então}}
\algrenewcommand\algorithmicelse{\textbf{Senão}}
\algrenewcommand\algorithmicreturn{\textbf{Devolve}}

\makeatletter
\renewcommand{\ALG@name}{Algoritmo}
\makeatother
\renewcommand{\listalgorithmname}{Lista de Algoritmos}

\begin{algorithm}[ht]
\caption{Algoritmo de Euclides}\label{euclides}
\begin{algorithmic}[1]
    \Procedure{Euclides}{$a, b$}\Comment{MDC de $a$ e $b$}
        \State $r\gets a\bmod b$
        \While{$r\not=0$}\Comment{Se $r = 0$, está resolvido}
            \State $a\gets b$
            \State $b\gets r$
            \State $r\gets a\bmod b$
        \EndWhile
        \State \Return $b$\Comment{O MDC é $b$}
    \EndProcedure
\end{algorithmic}
\end{algorithm}


\subsection{Códigos de programas}

Seus sonhos de ter o seu código com destaque sintático, avanços de tabulador e blocos de execução formatados automaticamente acaba de se tornar realidade! 

Vamos usar o ambiente \textit{minted}. Ele só precisa receber o nome da linguagem que você quer usar. O resto é com ele. Que tal?

\begin{minted}{python}
import numpy as np
    
def incmatrix(genl1, genl2):
    m = len(genl1)
    n = len(genl2)
    M = None  # To become the incidence matrix
    VT = np.zeros((n*m, 1), int)  # Dummy variable
    
    M1 = bitxormatrix(genl1)
    M2 = np.triu(bitxormatrix(genl2), 1) 

    for i in range(m-1):
        for j in range(i+1, m):
            [r, c] = np.where(M2 == M1[i, j])
            for k in range(len(r)):
                VT[(i)*n + r[k]] = 1
                VT[(i)*n + c[k]] = 1
                VT[(j)*n + r[k]] = 1
                VT[(j)*n + c[k]] = 1
                
                if M is None:
                    M = np.copy(VT)
                else:
                    M = np.concatenate((M, VT), 1)
                
                VT = np.zeros((n*m,1), int)
    return M
\end{minted}


\subsection{Redes neurais}

Bem sabemos que os diagramas que exibem arquiteturas de rede podem ser bastante complexos. Para desenhos sofisticados, o mais fácil é usar um dos muitos editores disponíveis. O desenho pronto pode ser exportado como arquivo e incluído no seu documento do \LaTeX{}. Alguns editores, como o GeoGebra, permitem que você exporte o desenho num dos formatos vetoriais, como TikZ, que podem ser incorporados diretamente no corpo do documento.

Para fazer redes neurais simples, existe o pacote \textit{neuralnetwork}, que permite implementar o ambiente de mesmo nome. Com ele, é possível descrever as camadas de uma rede \textit{Feed Forward} e as ligações entre elas. Observe que o comando para conectar os neurônios (\verb|\linklayers|) é dado \textit{depois} da declaração da camada.

\bigskip

\begin{center}  % Este é o ambiente de centralização. Pode ser usado com quase tudo.
\begin{neuralnetwork}[height=5]  % Use o tamanho da maior camada aqui (incluindo bias)
    \newcommand{\x}[2]{$x_#2$}  % Cria um novo comando \x
	\newcommand{\y}[2]{$\hat{y}_#2$}  
	\newcommand{\hfirst}[2]{\small $h^{(1)}_#2$}
	\newcommand{\hsecond}[2]{\small $h^{(2)}_#2$}
	
	% Define as camadas da rede
	\inputlayer[count=3, bias=false, title=Camada de\\entrada, text=\x]
	\hiddenlayer[count=4, bias=true, title=Camada\\escondida 1, text=\hfirst] \linklayers
	\hiddenlayer[count=3, bias=false, title=Camada\\escondida 2, text=\hsecond] \linklayers
	\outputlayer[count=2, title=Camada de\\saída, text=\y] \linklayers
\end{neuralnetwork}
\end{center}

    \chapter{Resultados e Discussão}
\label{cap:Resultados}

Neste capítulo devem ser exibidos e discutidos os resultados obtidos pela sua pesquisa.


\section{Descrição dos achados}
\label{sec:Descricao}

Trata-se da apresentação dos dados com base em três recursos:
    
\begin{itemize}
    \item Tabelas
    \item Gráficos
    \item Textos explicativos
\end{itemize}
    
Nem todos os resultados merecem aparecer no capítulo de resultados. Apresente somente os mais relevantes, pensando em manter o leitor informado sobre o que interessa --- e sem perder o foco do que interessa. Deixe de fora, por exemplo, resultados relativos à interminável série de tentativas de parametrização das redes neurais.


\section{Discussão}
\label{sec:Discussao}

Discutir os resultados significa, essencialmente, chamar a atenção para aqueles que sejam os mais importantes na sua pesquisa. Há duas formas básicas de se fazer isso:
    
\begin{enumerate}
    \item Você pode contrastar diferentes resultados alcançados pelo uso de diferentes modelos, por exemplo. 
    
    \item Comparar seus resultados com os de outras publicações. Use tabelas para facilitar a comparação e coloque os números do estado da arte em negrito. É preciso expor as possíveis razões da diferença: os outros autores usaram outros modelos? Outros dados?
\end{enumerate}


\section{Figuras e tabelas}

Vale a pena refrescar a memória sobre como trabalhar com figuras e tabelas no \LaTeX{}.

Sobre as figuras, embora o \LaTeX{} seja muito versátil na geração de desenhos vetoriais, não vamos tratar disso aqui. Você pode procurar saber mais a respeito procurando no Google ou na Wikibooks pelo pacote \textit{TikZ}.

As figuras de que vamos falar são arquivos externos incorporados ao documento. Os gráficos para apresentação de resultados são figuras desse tipo. É simples incorporar figuras ao texto. Elas devem ser incluídas em ambientes \textit{Figure}:

\begin{figure}[ht] % [ht] é a ordem de localização: primeiro "aqui" (here), e, se não der certo, no alto da página (top)
    \centering
    \includegraphics[width=0.4\linewidth]{./figuras/latex-logo.png}
    \caption{Logo oficial do \LaTeX{}.}
    \label{fig:logo}
\end{figure}


As tabelas serão feitas em duas etapas. A primeira é a criação de um ambiente de tabulações, \textit{tabular}, que cria os espaçamentos entre as células. A segunda é a colocação do \textit{tabular} num ambiente \textit{table}, que é flutuante, assim como o ambiente \textit{figure} --- ou seja, ele escolhe a melhor posição na página para não atrapalhar o fluxo de texto.

\begin{table}[ht]
\centering

\begin{tabular}{|l|c|c|}  % Alinhamento das células e desenho das linhas verticais
\hline  % Desenha uma linha horizontal
\textbf{Cabeçalho 1} & \textbf{Cab. 2} & \textbf{Cab. 3} \\
\hline\hline
Linha 1 & Coluna 1 & Coluna 2 \\
\hline
Linha 2 & Coluna 1 & Coluna 2 \\
\hline
\end{tabular}

\caption{Um exemplo de tabela.}
\label{tab:exemplo}
\end{table}

    \chapter{Conclusão}
\label{cap:Conclusao}

Este capítulo deve ser tão breve e sintético quanto possível. Procure retomar os principais achados de sua pesquisa e situá-los no panorama geral do problema encontrado na literatura, a fim de responder a estas questões: 

\begin{itemize}
    \item Qual a principal contribuição do seu trabalho para os estudos desse tema?
    
    \item Em que medida seus resultados podem ser generalizados para a análise de outros dados? Seria frustrante pensar que seu modelo só funciona com o conjunto de dados específico que você usou.
    
    \item Quais as limitações do seu trabalho? Não diga somente que você teria resultados melhores com mais dados ou com mais poder computacional. Fale das possíveis limitações ligadas à metodologia usada.
    
    \item Como você melhoraria seus resultados no futuro? Que outros métodos seria interessante pesquisar e por quê?
\end{itemize}



\clearpage

%%%%%%%%%%%%%%%%%%%%
\backmatter
    \pagestyle{plain}  % Sem cabeçalho
    
    \printbibliography[title={Referências}]

    % Apêndices (opcionais). Edite da forma desejada.
    \input{./pos-textual/Apendices.tex}
    
    % Anexos (opcionais). Edite da forma desejada.
    \input{./pos-textual/Anexos.tex}


\end{document}

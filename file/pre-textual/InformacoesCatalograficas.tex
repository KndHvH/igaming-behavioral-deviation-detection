%%%%%%%%%%%%%%%%%%%%%%%%%%%%%%%%%%%%%%%%%%%%%%%%%%%%%%%%%%%%%%
% PREENCHA ABAIXO COM AS INFORMAÇÕES NECESSÁRIAS
%%%%%%%%%%%%%%%%%%%%%%%%%%%%%%%%%%%%%%%%%%%%%%%%%%%%%%%%%%%%%%

%%% Título 
\newcommand{\TítuloDoTrabalho}{%
  Título do trabalho apenas com a primeira letra em maiúscula, sem ponto final}

%%% Título em inglês
\newcommand{\TítuloEmIngles}{%
  Título do trabalho (em inglês) apenas com a primeira letra
  em maiúscula, sem ponto final}

%%% Versão original ou corrigida? Escolha.
\newcommand{\Versao}{Versão 
Original
% Corrigida
}

%%% Autor (nome completo por extenso)
\newcommand{\Autor}{%
  Nome completo do autor 
}

%%% Qual é seu último nome? (Como você usa nas publicações). Se houver
%%% Júnior, Filho, etc, indique apropriadamente. Ex: Silva Filho.
\newcommand{\Sobrenome}{%
  Tal}

%%% Qual é seu primeiro nome, nome do meio, e demais partes do
%%% sobrenome (não incluídas acima)? A soma deste item e do anterior
%%% devem formar seu nome completo por extenso. Não abrevie.
\newcommand{\Nome}{%
  Fulano de}

%%% Escreva as iniciais do seu nome *sem o sobrenome* com a pontuação correspondente,
%%% tal como você quer ver na Ficha Catalográfica
\newcommand{\IniciaisDoNome}{%
  F.}


%%%% Indique o nome do seu Orientador (nome completo por extenso)
\newcommand{\Orientador}{%
  Nome completo do orientador}

%%% Seu orientador(a) é Prof. Dr. ou Profa. Dra.? Selecione umas das
%%% linhas abaixo. Não remova as barras invertidas, apenas selecione a
%%% linha apropriada, comentando a outra
\newcommand{\DoutorOuDoutora}{%
  Prof.\ Dr.\
  %Profa.\ Dra.\
}

%%%% Dizeres da titulação
\newcommand{\TítuloObtido}{%
  Monografia apresentada ao Programa de Educação Continuada em Engenharia da Escola Politécnica da Universidade de São Paulo como parte dos requisitos para conclusão do curso de Especialização em Inteligência Artificial.}


%%%% Informe o ano de depósito do trabalho
\newcommand{\AnoDepósito}{%
  2021}

%%%% Indique o número total de páginas do trabalho
\newcommand{\NumPáginas}{%
  150}

%%%% Tipo de trabalho
\newcommand{\TipoTrabalho}{%
  Monografia (Especialização em Inteligência Artificial)}


%%%% Indique as palavras-chave para a FICHA CATALOGRÁFICA
% Cada uma delas deve ser precedida por um número com ponto.
% Todas palavras devem iniciar com letras maiúsculas
\newcommand{\PalavrasChaveFicha}{%
  1.~Palavra-Chave1 ~~2.~Palavra-Chave2 ~~3.~Palavra-Chave3}


%%% Palavras chave para Resumo e Abstract

% Resumo
% Use a mesma lista indicada acima
% Todas palavras devem iniciar com letras maiúsculas e terminar com um ponto final.
\newcommand{\PalavrasChave}{%
  Palavra-Chave1. Palavra-Chave2. Palavra-Chave3.}

%%%% Palavras-chave em inglês.
% Siga as regras para as palavras-chave em português.
\newcommand{\Keywords}{%
  Keyword1. Keyword2. Keyword3,}

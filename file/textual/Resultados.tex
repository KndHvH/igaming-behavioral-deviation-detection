\chapter{Resultados e Discussão}
\label{cap:Resultados}

Neste capítulo devem ser exibidos e discutidos os resultados obtidos pela sua pesquisa.


\section{Descrição dos achados}
\label{sec:Descricao}

Trata-se da apresentação dos dados com base em três recursos:
    
\begin{itemize}
    \item Tabelas
    \item Gráficos
    \item Textos explicativos
\end{itemize}
    
Nem todos os resultados merecem aparecer no capítulo de resultados. Apresente somente os mais relevantes, pensando em manter o leitor informado sobre o que interessa --- e sem perder o foco do que interessa. Deixe de fora, por exemplo, resultados relativos à interminável série de tentativas de parametrização das redes neurais.


\section{Discussão}
\label{sec:Discussao}

Discutir os resultados significa, essencialmente, chamar a atenção para aqueles que sejam os mais importantes na sua pesquisa. Há duas formas básicas de se fazer isso:
    
\begin{enumerate}
    \item Você pode contrastar diferentes resultados alcançados pelo uso de diferentes modelos, por exemplo. 
    
    \item Comparar seus resultados com os de outras publicações. Use tabelas para facilitar a comparação e coloque os números do estado da arte em negrito. É preciso expor as possíveis razões da diferença: os outros autores usaram outros modelos? Outros dados?
\end{enumerate}


\section{Figuras e tabelas}

Vale a pena refrescar a memória sobre como trabalhar com figuras e tabelas no \LaTeX{}.

Sobre as figuras, embora o \LaTeX{} seja muito versátil na geração de desenhos vetoriais, não vamos tratar disso aqui. Você pode procurar saber mais a respeito procurando no Google ou na Wikibooks pelo pacote \textit{TikZ}.

As figuras de que vamos falar são arquivos externos incorporados ao documento. Os gráficos para apresentação de resultados são figuras desse tipo. É simples incorporar figuras ao texto. Elas devem ser incluídas em ambientes \textit{Figure}:

\begin{figure}[ht] % [ht] é a ordem de localização: primeiro "aqui" (here), e, se não der certo, no alto da página (top)
    \centering
    \includegraphics[width=0.4\linewidth]{./figuras/latex-logo.png}
    \caption{Logo oficial do \LaTeX{}.}
    \label{fig:logo}
\end{figure}


As tabelas serão feitas em duas etapas. A primeira é a criação de um ambiente de tabulações, \textit{tabular}, que cria os espaçamentos entre as células. A segunda é a colocação do \textit{tabular} num ambiente \textit{table}, que é flutuante, assim como o ambiente \textit{figure} --- ou seja, ele escolhe a melhor posição na página para não atrapalhar o fluxo de texto.

\begin{table}[ht]
\centering

\begin{tabular}{|l|c|c|}  % Alinhamento das células e desenho das linhas verticais
\hline  % Desenha uma linha horizontal
\textbf{Cabeçalho 1} & \textbf{Cab. 2} & \textbf{Cab. 3} \\
\hline\hline
Linha 1 & Coluna 1 & Coluna 2 \\
\hline
Linha 2 & Coluna 1 & Coluna 2 \\
\hline
\end{tabular}

\caption{Um exemplo de tabela.}
\label{tab:exemplo}
\end{table}
